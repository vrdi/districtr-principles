\documentclass{mgggarticle}
\usepackage[utf8]{inputenc}
\usepackage[english]{babel}
\usepackage{hyperref}

\title{Draft: Districtr Guiding Principles}
\author{Prepared by Natasha Dhamankar, Jane Hood, and Christopher Gernon\\ Voting Rights Data Institute, 2019}

\version{1.0}

\begin{document}

\begin{titlepage}

\maketitle



\newpage

\subsection*{Introduction}

The Metric Geometry and Gerrymandering Group (MGGG) is a Boston-based working group. Its mission is to study applications of geometry and computing to U.S. redistricting. Districtr and MGGG are non-partisan and not for profit and do not prioritize the agenda of any political party, organization, or business.\\

\noindent MGGG believes that gerrymandering of all kinds is a fundamental threat to democracy. Districtr is a project of MGGG that puts the tools of redistricting into the hands of the public by providing free, open-source software for redistricting. The tool allows users to understand how districting and gerrymandering impact themselves and the world around them.\\ 

\noindent We acknowledge that our technology, research, and education relating to redistricting pose ethical questions. Development of Districtr and other redistricting software should be guided by a set of principles, in order to ensure these tools are used to promote justice and equity. Below, we outline these values and discuss their implications for users and future software development.\\



\end{titlepage}


\decoratedsection{Districtr User Guiding Principles}

\begin{enumerate}
  \item Educate users about the redistricting process and help users understand and improve their own representation.
  \item Promote civic education and research. 
  \begin{itemize}
      \item We created Districtr to provide an easy and enjoyable tool for learning about the redistricting process and exploring possible districting plans. MGGG collaborates with educators to use Districtr as a teaching resource.  We also use Districtr to research and analyze possible districting plans in our legal and consulting work. 
  \end{itemize}
  \item Prioritize community knowledge

  

  \begin{itemize}
      \item We believe that people deserve a voice in how their districts are created. We hope that Districtr’s maps provide users helpful information to guide their creation of districting plans. We plan to expand the tool to represent redistricting and communities beyond cartography and line drawing, such as through mental maps. We welcome and encourage feedback about what you’d like to see on Districtr.

  \end{itemize}
  \item Prioritize accessibility to users with various educational and language backgrounds, technological resources, abilities, and identities. 
  \begin{itemize}
      \item We are currently building our accessibility tools. Please contact us at max@mggg.org to request accessible information/services.
  \end{itemize}
  \item Protect user privacy
  \begin{itemize}
      \item Districtr currently does not collect any of your personal data while using the website. We do not use any cookies, third party trackers or analytics. If you sign up for an account with Districtr, we record your email into our database; however, this will not be used for any communications without your consent. You can delete your account at anytime via this link.
  \end{itemize}
  \item Develop transparent and open-source tools in order to promote public access to analytical resources about redistricting. 

\end{enumerate}

\newpage
\decoratedsection{Districtr Software Development Guiding Principles}

\begin{enumerate}
  \item Educate users about the redistricting process and help users understand and improve their own representation.
  \item Promote civic education and research.
  \begin{itemize}
      \item Assume no prior knowledge from user: clearly explain or provide resources about how certain functions, tools, and laws work. 
      \item Collaborate with educators, students, and academics to improve Districtr as a teaching tool and include it in curricula.
      \item Use of districtr should be easy and enjoyable for the user.
      \item No third party advertising or marketing on the site. 
      \item Balance development priorities for research and educational purposes.
  \end{itemize}
  \item Prioritize community knowledge; this tool is simply a way to visualize communities.
  \begin{itemize}
      \item Design and test software development to minimize potential for harm and negative outcomes resulting from this tool. 
      \item Districtr modules should be motivated by collaborations with groups (such as communities and civil rights groups) in that particular state. 
      \item Center the opinions and needs of local community groups and organizations in the development process. Respond to changing needs. 
      \item Undergo user testing of Districtr modules from within the given community while in the design and development phase.
      \item Engage with the principles of critical GIS.
      \item Develop tools to include different perspectives beyond line-drawing and maps, such as photos, art, and mental maps. 
  \end{itemize}
  \item Prioritize accessibility to users with various educational and language backgrounds, technological resources, abilities, and identities. 
  \begin{itemize}
      \item Design so that people with minimal prior knowledge on districting or American government can interact with this tool.
      \item Ensure that content is accessible to a wider range of people with disabilities. 
      \begin{itemize}
          \item Follow standards, such as the Web Content Accessibility Guidelines (linked here: \url{https://w3c.github.io/wcag/21/guidelines/}) as a framework to include accommodations “for blindness and low vision, deafness and hearing loss, limited movement, speech disabilities, photosensitivity, and combinations of these, and some accommodations for learning disabilities and cognitive limitations.” 
          \item For instance, provide a text equivalent for all non-text elements. On the web page, all information conveyed with color should also be conveyed without color. 
          \item Plan for improvement of the website in order to ensure that it meets accessibility standards. 
          \item Include contact information (email address or phone number) so that visitors can request accessible information/services.

      \end{itemize}
  \end{itemize}
  \item Protect user privacy
  \begin{itemize}
      \item Respect the highest standard of privacy regulations and guidelines. At the time of publishing, this is the 2018 California Consumer Privacy Act: 
      \begin{itemize}
          \item Users have a right to know what personal information and data (eg. plan proposals) are being collected, access it, and request that it be deleted. 
          \item Users can opt-in \footnote{ Note: this is a difference from the CCPA, which only requires users to be able to opt-out.} to the sharing of their personal information and data and must know who their information is being shared with.
          \item Users have equal service regardless of how users choose to exercise their privacy rights.


      \end{itemize}
  \end{itemize}
  \item Develop transparent and open-source tools in order to promote public access to analytical resources about redistricting. 
  \begin{itemize}
      \item Ensure that all data used for the tool are freely available and well-documented.
      \item Declare the source of the data, its processing, and our thoughts on its quality.
  \end{itemize}

\end{enumerate}

\end{document}
